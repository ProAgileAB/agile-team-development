%%
% Copyright (c) 2018, Pascal Wagler;  
% Copyright (c) 2014--2018, John MacFarlane
% 
% All rights reserved.
% 
% Redistribution and use in source and binary forms, with or without 
% modification, are permitted provided that the following conditions 
% are met:
% 
% - Redistributions of source code must retain the above copyright 
% notice, this list of conditions and the following disclaimer.
% 
% - Redistributions in binary form must reproduce the above copyright 
% notice, this list of conditions and the following disclaimer in the 
% documentation and/or other materials provided with the distribution.
% 
% - Neither the name of John MacFarlane nor the names of other 
% contributors may be used to endorse or promote products derived 
% from this software without specific prior written permission.
% 
% THIS SOFTWARE IS PROVIDED BY THE COPYRIGHT HOLDERS AND CONTRIBUTORS 
% "AS IS" AND ANY EXPRESS OR IMPLIED WARRANTIES, INCLUDING, BUT NOT 
% LIMITED TO, THE IMPLIED WARRANTIES OF MERCHANTABILITY AND FITNESS 
% FOR A PARTICULAR PURPOSE ARE DISCLAIMED. IN NO EVENT SHALL THE 
% COPYRIGHT OWNER OR CONTRIBUTORS BE LIABLE FOR ANY DIRECT, INDIRECT, 
% INCIDENTAL, SPECIAL, EXEMPLARY, OR CONSEQUENTIAL DAMAGES (INCLUDING,
% BUT NOT LIMITED TO, PROCUREMENT OF SUBSTITUTE GOODS OR SERVICES; 
% LOSS OF USE, DATA, OR PROFITS; OR BUSINESS INTERRUPTION) HOWEVER 
% CAUSED AND ON ANY THEORY OF LIABILITY, WHETHER IN CONTRACT, STRICT 
% LIABILITY, OR TORT (INCLUDING NEGLIGENCE OR OTHERWISE) ARISING IN 
% ANY WAY OUT OF THE USE OF THIS SOFTWARE, EVEN IF ADVISED OF THE 
% POSSIBILITY OF SUCH DAMAGE.
%%

%%
% For usage information and examples visit the GitHub page of this template:
% https://github.com/Wandmalfarbe/pandoc-latex-template
%%

\PassOptionsToPackage{unicode=true}{hyperref} % options for packages loaded elsewhere
\PassOptionsToPackage{hyphens}{url}
\PassOptionsToPackage{dvipsnames,svgnames*,table}{xcolor}
%
\documentclass[a4paper,,tablecaptionabove]{scrartcl}
\usepackage{lmodern}
\usepackage{setspace}
\setstretch{1.2}
\usepackage{amssymb,amsmath}
\usepackage{ifxetex,ifluatex}
\usepackage{fixltx2e} % provides \textsubscript
\ifnum 0\ifxetex 1\fi\ifluatex 1\fi=0 % if pdftex
  \usepackage[T1]{fontenc}
  \usepackage[utf8]{inputenc}
  \usepackage{textcomp} % provides euro and other symbols
\else % if luatex or xelatex
  \usepackage{unicode-math}
  \defaultfontfeatures{Ligatures=TeX,Scale=MatchLowercase}
\fi
% use upquote if available, for straight quotes in verbatim environments
\IfFileExists{upquote.sty}{\usepackage{upquote}}{}
% use microtype if available
\IfFileExists{microtype.sty}{%
\usepackage[]{microtype}
\UseMicrotypeSet[protrusion]{basicmath} % disable protrusion for tt fonts
}{}
\IfFileExists{parskip.sty}{%
\usepackage{parskip}
}{% else
\setlength{\parindent}{0pt}
\setlength{\parskip}{6pt plus 2pt minus 1pt}
}
\usepackage{hyperref}
\hypersetup{
            pdfborder={0 0 0},
            breaklinks=true}
\urlstyle{same}  % don't use monospace font for urls
\usepackage[margin=2.5cm,includehead=true,includefoot=true,centering]{geometry}
\setlength{\emergencystretch}{3em}  % prevent overfull lines
\providecommand{\tightlist}{%
  \setlength{\itemsep}{0pt}\setlength{\parskip}{0pt}}
\setcounter{secnumdepth}{0}
% Redefines (sub)paragraphs to behave more like sections
\ifx\paragraph\undefined\else
\let\oldparagraph\paragraph
\renewcommand{\paragraph}[1]{\oldparagraph{#1}\mbox{}}
\fi
\ifx\subparagraph\undefined\else
\let\oldsubparagraph\subparagraph
\renewcommand{\subparagraph}[1]{\oldsubparagraph{#1}\mbox{}}
\fi

% Make use of float-package and set default placement for figures to H
\usepackage{float}
\floatplacement{figure}{H}


\date{}





%%
%% added
%%

%
% No language specified? take American English.
%

\ifnum 0\ifxetex 1\fi\ifluatex 1\fi=0 % if pdftex
  \usepackage[shorthands=off,main=english]{babel}
\else
    % See issue https://github.com/reutenauer/polyglossia/issues/127
  \renewcommand*\familydefault{\sfdefault}
    % load polyglossia as late as possible as it *could* call bidi if RTL lang (e.g. Hebrew or Arabic)
  \usepackage{polyglossia}
  \setmainlanguage[]{english}
\fi


%
% colors
%
\usepackage[]{xcolor}

%
% listing colors
%
\definecolor{listing-background}{HTML}{F7F7F7}
\definecolor{listing-rule}{HTML}{B3B2B3}
\definecolor{listing-numbers}{HTML}{B3B2B3}
\definecolor{listing-text-color}{HTML}{000000}
\definecolor{listing-keyword}{HTML}{435489}
\definecolor{listing-identifier}{HTML}{435489}
\definecolor{listing-string}{HTML}{00999A}
\definecolor{listing-comment}{HTML}{8E8E8E}
\definecolor{listing-javadoc-comment}{HTML}{006CA9}

%\definecolor{listing-background}{rgb}{0.97,0.97,0.97}
%\definecolor{listing-rule}{HTML}{B3B2B3}
%\definecolor{listing-numbers}{HTML}{B3B2B3}
%\definecolor{listing-text-color}{HTML}{000000}
%\definecolor{listing-keyword}{HTML}{D8006B}
%\definecolor{listing-identifier}{HTML}{000000}
%\definecolor{listing-string}{HTML}{006CA9}
%\definecolor{listing-comment}{rgb}{0.25,0.5,0.35}
%\definecolor{listing-javadoc-comment}{HTML}{006CA9}

%
% for the background color of the title page
%

%
% TOC depth and 
% section numbering depth
%
\setcounter{tocdepth}{3}

%
% break urls
%
\PassOptionsToPackage{hyphens}{url}

%
% When using babel or polyglossia with biblatex, loading csquotes is recommended 
% to ensure that quoted texts are typeset according to the rules of your main language.
%
\usepackage{csquotes}

%
% captions
%
\definecolor{caption-color}{HTML}{777777}
\usepackage[font={stretch=1.2}, textfont={color=caption-color}, position=top, skip=4mm, labelfont=bf, singlelinecheck=false, justification=raggedright]{caption}
\setcapindent{0em}
\captionsetup[longtable]{position=above}

%
% blockquote
%
\definecolor{blockquote-border}{RGB}{221,221,221}
\definecolor{blockquote-text}{RGB}{119,119,119}
\usepackage{mdframed}
\newmdenv[rightline=false,bottomline=false,topline=false,linewidth=3pt,linecolor=blockquote-border,skipabove=\parskip]{customblockquote}
\renewenvironment{quote}{\begin{customblockquote}\list{}{\rightmargin=0em\leftmargin=0em}%
\item\relax\color{blockquote-text}\ignorespaces}{\unskip\unskip\endlist\end{customblockquote}}

%
% Source Sans Pro as the de­fault font fam­ily
% Source Code Pro for monospace text
%
% 'default' option sets the default 
% font family to Source Sans Pro, not \sfdefault.
%
\usepackage[default]{sourcesanspro}
\usepackage{sourcecodepro}

%
% heading color
%
\definecolor{heading-color}{RGB}{40,40,40}
\addtokomafont{section}{\color{heading-color}}
% When using the classes report, scrreprt, book, 
% scrbook or memoir, uncomment the following line.
%\addtokomafont{chapter}{\color{heading-color}}

%
% variables for title and author
%
\usepackage{titling}
\title{}
\author{}

%
% tables
%

%
% remove paragraph indention
%
\setlength{\parindent}{0pt}
\setlength{\parskip}{6pt plus 2pt minus 1pt}
\setlength{\emergencystretch}{3em}  % prevent overfull lines

%
%
% Listings
%
%


%
% header and footer
%
\usepackage{fancyhdr}
\pagestyle{fancy}
\fancyhead{}
\fancyfoot{}
%\lhead[]{}
\lhead[]{}
\chead[]{}
\rhead[]{}
\lfoot[\thepage]{}
\cfoot[]{}
\rfoot[]{\thepage}
\renewcommand{\headrulewidth}{0.4pt}
\renewcommand{\footrulewidth}{0.4pt}

%%
%% end added
%%

\begin{document}

%%
%% begin titlepage
%%


%% Start customization for 
%% Proagile

\fancyhead[CO,CE]{This is fancy header}
\fancyfoot[CO,CE]{And this is a fancy footer}
\fancyfoot[LE,RO]{\thepage}
\fancypagestyle{plain}{\pagestyle{fancy}}


\renewcommand{\headrulewidth}{0pt}
\renewcommand{\footrulewidth}{0.4pt}

\pagestyle{fancy}
\fancyhf{
  \cfoot{\tiny{Shared 2018 by ProAgile AB under \href{https://creativecommons.org/licenses/by-sa/4.0/}{Creative Commons Attribution ShareAlike 4.0 International license}}}}
  \rfoot{\thepage}

%% End customization for 
%% Proagile

%%
%% end titlepage
%%


\hypertarget{product-goals-for-teams}{%
\section*{Product Goals for Teams}\label{product-goals-for-teams}}
\addcontentsline{toc}{section}{Product Goals for Teams}

\hypertarget{time-required}{%
\subsection*{Time required}\label{time-required}}
\addcontentsline{toc}{subsection}{Time required}

Typically 30 min - 2 hours, depending on how unfamiliar/unclear the
vision and goals are to the team

\hypertarget{materials-required}{%
\subsection*{Materials Required}\label{materials-required}}
\addcontentsline{toc}{subsection}{Materials Required}

\begin{itemize}
\tightlist
\item
  Orange \& green positits
\item
  Markers
\end{itemize}

\hypertarget{purpose}{%
\subsection*{Purpose}\label{purpose}}
\addcontentsline{toc}{subsection}{Purpose}

This is a teamstart/team development exercise. The purpose of it is for
the team to get to know their purpose more in detail, for them to
process it, discuss it, understand it and internalize it.

Having a clear, compelling purpose is the most important factor when it
comes to how a team performs. With no shared goal there will no team.

For a team we usually cover goals from several perspectives:

\begin{itemize}
\item
  Organization

  \begin{itemize}
  \tightlist
  \item
    \textbf{Product ← covered in this guide}
  \end{itemize}
\item
  Team
\item
  Individual
\end{itemize}

\hypertarget{preparations}{%
\subsection*{Preparations}\label{preparations}}
\addcontentsline{toc}{subsection}{Preparations}

Book some prep-sessions with some product manager/PO that is most
relevant for the team. They need to present the vision/goals from
product point of view. Expect at least 1+2 hours prep with some days in
between to create attractive visions/a compelling direction from product
point of view.

The purpose of the PO participating in this session is: - To present
long term (and medium term) goals in a way so that the team understands
context, actually knows what to do, what is important and why it is
important. - The above is needed to enable the team to make god
decentralized decisions

\begin{itemize}
\tightlist
\item
  To energize and motivate the team by describing the purpose in a way
  that is attractive to them,
\end{itemize}

During this prep session, focus on how the goal/visio can be made
attractive to the team.

Here are some suggestion on how to do that:

\begin{itemize}
\item
  Avoid powerpoints. Human to human communication is most often best
  done without it.
\item
  A personal story is one of the most effective ways of communicating a
  vision. This comes from 100 000 years of human history telling around
  campfires before we invented more elaborate writing, so it is a
  uniquely tested and effective way to convey information and
  motivate/energize others
\item
  The best ways in general to formulate a direction/goal is to focus
  around what good will we do in the world. Whose life will be better by
  our efforts. This could also be formulated using stories about current
  (bad) situations from a human point of view
\item
  The story is ideally end user/stakeholder/purpose centric but can can
  be about \enquote{how come the PO wanted to work with this?} "What
  makes him/her excited about the possibilities
\item
  Wording and expressions should be personal and use emotion. Words like
  proud, excited, sad etc are good to use. We do not want dry
  \enquote{professional} language.\\
\item
  A story could also be imaginary, about the future. There is one
  example that I remember vividly, among other things the message
  included: \enquote{Think about next summer at the convention, I would
  like us to be called upon the stage to receive the reward for most
  innovative solution in the healthcare business}. In this example I
  remember that the product manager also were very emotional speaking
  about the hard works spent on a bit boring stuff during the year and
  how they finally was time to focus forward and on innovation. I also
  remember that her story started out with her walking outside in the
  morning. Crispness of the air and other details was included in good
  storyteller fashion. It was hugely successful as a vision and people
  kept coming back to it to make sure they stayed on target: \enquote{Is
  this the best way for us to win that reward next year}
\end{itemize}

\hypertarget{how}{%
\subsection*{How}\label{how}}
\addcontentsline{toc}{subsection}{How}

\begin{itemize}
\item
  \emph{The purpose of the next session is for you to discuss the long
  goals and direction of the team with our PO/PM. And to create your own
  summary of it.}
\item
  \emph{Btw, lets do a quick pairwise discussion: \enquote{Why would it
  makes sense for you all to know about the goals on a higher level and
  not only get task by task in the print panning?} - 2 minutes pairwise
  discussion}
\item
  Debrief by asking a few pairs. Make sure answers include that self
  organized teams need to understand the context and purpose to make
  good everyday decisions.
\item
  It is actually also not so uncommon for people/projects to actually
  not now the goal of projects and thereby waste a lot of time. I know
  one project that spent 6 month on porting a UI to Silverlight while
  the purpose of the project had nothing to do with that.
\item
  \emph{Our PO will now present his/her ideas. While talking you all
  will write some comments on orange and green positits. Green postits
  for clear goals/directions that you hear. Orange for items you feel
  could be more clear or that you would like to discuss}
\item
  Let the PO do the presentation. After, collect all clear/unclear
  postits on some flipcharts. Facilitate a summary of the green ones and
  a discussion/clarification of orange ones together with the PO/PM
\item
  Save the physical results for use in later retrospectives and other
  sessions
\end{itemize}

\end{document}

